\section{Introduction}
\label{sec:introduction}
Driftworld is a Unity project implemented for use in the Unity editor. The entirety of interactivity is within the editor GUI and the project has no meaningful executable scene. Any feedback is in a console log and the state of the planet is observed within the static scene rendering. This is the most obvious difference from the implementation in the original article from which Driftworld draws inspiration - simulation described in the article offers interactivity while the simulation is running \cite{cortial}.

The workflow follows Cortial et al. in a lot of details, although experience and chosen software tools pose several restrictions. At first a Delaunay triangulation mesh is imported from prepared binary files. Then a set of tectonic plates is created by partitioning said mesh. The planet evolution is performed in repeated tectonic steps. Every step deals with plate subduction, possible continental collision, new crust creation because of diverging ocean plates. slab pull due to subduction influence, erosion and crust damping, and finally, rifting plates. At any time the current state can be saved as a binary file for further use. This follows the original article \cite{cortial}. Driftworld, however, differs in two rather important steps: continental collisions are always plate-wide and plate rifting follows somewhat different probability mechanics. This is mainly for fine-tuning and can change in future updates, as these changes further simplify an already simplified model and were done as a saving grace from implementation difficulties.

The output of Driftworld is binary planet data with varying resolution of sphere sampling. Provided data is crust age, elevation, plate assignment, crust thickness and orogeny. The binary file keeps the original topology for easier manipulation. The user can also at any time export the current texture overlay as a PNG image.

This documentation serves both as a user's manual and a quick introduction into the problematic. Section \ref{sec:geometry} defines basic terms, mathematical objects and their properties. Section \ref{sec:tectonicmodel} follows with details of the used simplified tectonic model. The actual implementation with neccessary details and context are discussed in the section \ref{sec:implementation}. As an important part, performance of the simulation and related issues are the topic of section \ref{sec:performance}. We conclude the status quo of the project in section \ref{sec:conclusion}.

\subsection{Motivation}
\label{subsec:motivation}
Procedural terrain generation is an important part for a number of computer games \cite{generatedgames}. Usually, these games employ random generators to increase variety on a theme, such as a map layout. Indeed, in my subjective opinion a~player's experience is greatly enriched by variety, especially in the game environment. This comes with an apparent caveat that purely procedural generation may lack the sense of creativity, leading to mundanely repeating patterns \cite{daggerfall}. 

With the onset of newer technologies (e. g. increased GPU power), we are able to perform more computationally-intensive tasks. When it comes to the terrain generation, even a regular user without access to high-tier hardware can try more sophisticated alternatives to simpler algorithms. Arguably, more realistic worlds bring the feeling of familiarity to the experience. If we can create a~more realistic, yet still random map/world/neighbourhood, the possibilities are endless.
\newpage
Following thoughts are purely my personal view. As a life-long video games fan, I have always gravitated towards story-telling games, especially those taking place in an open world. Among these, I'd like to mention Baldur's Gate series, The Elder Scrolls series, Might \& Magic series, Fallout 4 and Mass Effect series. At the same time, I have been also drawn to grand building games taking place in complex worlds. Transport Tycoon or Caesar III and its modern re-implementation Augustus \cite{augustus} were a heavy influence, lately Factorio or Rimworld. Rimworld stands apart in its uniqueness, as it can be understood rather as a story generator than a game \cite{rimworld}. In large part, the idea of Driftworld came from the works of J. R. R. Tolkien and watching the 1997-2007 Stargate series - the series' take on mythology context in human societies in particular.

Epic stories build on cohesion. In this regard, countless debates take place about details. It takes a~great deal of time to create viable environment to match an idea for a story, if that story is told over long periods of time. Driftworld aspires to one thing: help create a platform in which stories can take place. First step is this project -- to create a rough map. 
