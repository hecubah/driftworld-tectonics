\section{Conclusion}
\label{sec:conclusion}
The promise of Driftworld Tectonics is to create interesting planet surfaces either for rendering or further development -- a~goal which I believe it has achieved. The simulations loosely follow tectonic processes present on Earth. Since the project is an attempted implementation of the broadly described tectonic method of Cortial et al., it should provide similar features and advantages. Indeed, many of the features seem to be present (at least in my opinion). However, it should be noted that the work presented by Cortial et al. is much more sophisticated. Driftworld Tectonics, however, offers a~detailed description of its every aspect, which may inspire further improvements, optimizations and extensions. The project is meant to be as modular as possible so that changes would not break the project structure too badly.

The lack of terrain amplification is intentional to keep the project somewhat simple. All relevant data is available in the output files, so any subsequent amplification procedure may take place. It stands to reason that folding direction should be implemented in this stage of simulation -- hopefully, this will soon be the case.

There is still much work to be done. Issues that have come up during the development of Driftworld need to be addressed so that any simulation can be reliable and smoothly operating.

It is my hope that this project inspires content creators and developers to either improve on present tectonic approximation projects or make use of Driftworld Tectonics or other available software to create worlds and stories told on these worlds.
\subsection{Continuation of work}
Driftworld Tectonics is only the first step. The real ambition is to create rich worlds with water, weather, winter or spring, plants and animals, and ultimately, people. Each part of this ambition is a~huge project on its own, it has different requirements, different ideas and approaches. At this moment, I can hardly imagine the complexity of these projects and how monumental the effort will be to undertake them. I do believe that every part of this ambition has some value on its own.

For the immediate future, a~climate model should be created on the simulation output. Calculating heat distribution during a year and the hydrosphere model seems like a logical next step. Then, amplification of the terrain is necessary to obtain any kind of presentable planet.

Any notion of a~biosphere depends on the previous steps and in turn, influences both the climate and the hydrosphere as a~whole. A~stable model should follow where deserts have a~reason and forests do not thrive without rain. The goal is not to create true simulations but to bring believable biomes.

The final step -- people. This unfortunately seems so far that all ambition -- culture, language, history, agriculture, war or infrastructure -- is a~vague, shapeless dream, so far only in the imagination of a~writer. Maybe after a~few years, some parts of this ambition will take more solid shape so that anyone can be surprised to hear an exotic language spoken on a~tropical island in the middle of an ocean.