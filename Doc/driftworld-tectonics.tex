% !TEX TS-program = pdflatex
% !TEX encoding = UTF-8 Unicode

% This is a simple template for a LaTeX document using the "article" class.
% See "book", "report", "letter" for other types of document.


% ad-hoc changes to typesetting:
% -

\documentclass[11pt]{article} % use larger type; default would be 10pt

\usepackage[utf8]{inputenc} % set input encoding (not needed with XeLaTeX)

%%% Examples of Article customizations
% These packages are optional, depending whether you want the features they provide.
% See the LaTeX Companion or other references for full information.

%%% PAGE DIMENSIONS
\usepackage{geometry} % to change the page dimensions
\geometry{a4paper} % or letterpaper (US) or a5paper or....
\geometry{margin=20mm} % for example, change the margins to 2 inches all round
% \geometry{landscape} % set up the page for landscape
%   read geometry.pdf for detailed page layout information
\setlength{\parindent}{0em}
\setlength{\parskip}{2em}
\usepackage{graphicx} % support the \includegraphics command and options

% \usepackage[parfill]{parskip} % Activate to begin paragraphs with an empty line rather than an indent

%%% PACKAGES
\usepackage{booktabs} % for much better looking tables
\usepackage{array} % for better arrays (eg matrices) in maths
\usepackage{paralist} % very flexible & customisable lists (eg. enumerate/itemize, etc.)
\usepackage{verbatim} % adds environment for commenting out blocks of text & for better verbatim
%\usepackage{subfig} % make it possible to include more than one captioned figure/table in a single float
\usepackage{subcaption} % make it possible to include more than one captioned figure/table in a single float
% These packages are all incorporated in the memoir class to one degree or another...

%%% HEADERS & FOOTERS
\usepackage{fancyhdr} % This should be set AFTER setting up the page geometry
\pagestyle{fancy} % options: empty , plain , fancy
\renewcommand{\headrulewidth}{0pt} % customise the layout...
\lhead{}\chead{}\rhead{}
\lfoot{}\cfoot{\thepage}\rfoot{}

%%% SECTION TITLE APPEARANCE
\usepackage{sectsty}
\allsectionsfont{\sffamily\mdseries\upshape} % (See the fntguide.pdf for font help)
% (This matches ConTeXt defaults)

%%% ToC (table of contents) APPEARANCE
\usepackage[nottoc,section, numbib]{tocbibind} % Put the bibliography in the ToC
\usepackage[titles]{tocloft} % Alter the style of the Table of Contents
\renewcommand{\cftsecfont}{\rmfamily\mdseries\upshape}
\renewcommand{\cftsecpagefont}{\rmfamily\mdseries\upshape} % No bold!
\newcommand{\asd}[1]{\texttt{#1}}
%%% END Article customizations
\graphicspath{ {./resources/} }
\usepackage[hidelinks]{hyperref}
\usepackage{amsmath}
\usepackage{amsfonts}
\hypersetup{
    colorlinks=true,
    citecolor=[RGB]{0,0,200},
    linkcolor=[RGB]{128,0,0},
    filecolor=magenta,      
    urlcolor=cyan,
    pdftitle={Driftworld},
    pdfauthor={Adalbert Delong},
    }

\title{Driftworld Tectonics 1.0: an overview}
\author{Adalbert Delong}
\date{2nd August 2022}

\begin{document}
\maketitle
%\section*{Abstract}
Driftworld Tectonics is a project written for Unity editor, used to create basic forms of planets. It utilizes ideas and methods described in an article by Yann Cortial et al. in 2019 \cite{cortial}. Planets are created by~reading basic template topology, performing a simplified tectonic simulation of the planet's crust and~then exporting the raw data to customized binary files. Users can display various overlays of the~planet during the simulation, as well as see the surface elevation mesh.\par
This documentation describes some basic theoretical framework, details of the simulation and the implementation. Hopefully it can be of use to anyone interested in this topic, who either just wants to play around with Driftworld or build their own projects.\par
Project Driftworld is licensed under a \href{https://creativecommons.org/licenses/by/4.0/}{Creative Commons Attribution 4.0 International License}.
\section*{Acknowledgements}

I would like to thank Dr. Yann Cortial of the National Institute of Applied Sciences of Lyon for discussing his article. His answers to my various inquiries helped me decide the scope and form of Driftworld. I would also like to thank Dr. Daniel Meister of AMD Japan Co. Ltd. for his comments on the use of bounding volume hiearchy algorithms. Driftworld implements a part of an algorithm described in one of his publications \cite{meister}. More thanks to \href{https://www.prowaretech.com}{PROWAREtech} for allowing me to include an adapted version of an example implementation of Mersenne Twister random number generator in the project.

I would like to thank Ben Golus for his help with UV texture mapping and his advice on texturing. %href??

Special thanks to my friends Vilém and Matyáš and other members of our Discord server for discussing ideas and for their feedback.

The project was created using Unity Editor, lately in its version 2021.3.1f1. The C\# code is kept in a~MS Visual Studio Community 2022 project, image materials come from Unity Editor screenshots, Geogebra projects, Blender and Dia. Documentation uses \LaTeX\hspace*{0.3em}in its TeX Live implementation.
\section*{Disclaimer}

Over the past two years, Driftworld evolved both in terms of ideas and terms of implementation. The absolute majority of concepts beyond mathematics were completely new to me when the project started, so the code changed often and many times was almost completely rewritten as I learned. Some older parts remained which can cause a correct impression of inhomogeneity. I do not claim the implementation is flawless and although a lot of the shaky cases were accounted for, some unforseen mistakes may and probably do remain.

I would like to ask anyone using the project to tolerate possible mistakes. There is more work to be done and I would be grateful for feedback.

Thank you for your consideration.
\newpage
\tableofcontents
\newpage
\section{Introduction}
\label{sec:introduction}
Driftworld is a Unity project implemented for use in the Unity editor. The entirety of interactivity is within the editor GUI and the project has no meaningful executable scene. Any feedback is in a console log and the state of the planet is observed within the static scene rendering. This is the most obvious difference from the implementation in the original article from which Driftworld draws inspiration - simulation described in the article offers interactivity while the simulation is running \cite{cortial}.

The workflow follows Cortial et al. in a lot of details, although experience and chosen software tools pose several restrictions. At first a Delaunay triangulation mesh is imported from prepared binary files. Then a set of tectonic plates is created by partitioning said mesh. The planet evolution is performed in repeated tectonic steps. Every step deals with plate subduction, possible continental collision, new crust creation because of diverging ocean plates. slab pull due to subduction influence, erosion and crust damping, and finally, rifting plates. At any time the current state can be saved as a binary file for further use. This follows the original article \cite{cortial}. Driftworld, however, differs in two rather important steps: continental collisions are always plate-wide and plate rifting follows somewhat different probability mechanics. This is mainly for fine-tuning and can change in future updates, as these changes further simplify an already simplified model and were done as a saving grace from implementation difficulties.

The output of Driftworld is binary planet data with varying resolution of sphere sampling. Provided data is crust age, elevation, plate assignment, crust thickness and orogeny. The binary file keeps the original topology for easier manipulation. The user can also at any time export the current texture overlay as a PNG image.

This documentation serves both as a user's manual and a quick introduction into the problematic. Section \ref{sec:geometry} defines basic terms, mathematical objects and their properties. Section \ref{sec:tectonicmodel} follows with details of the used simplified tectonic model. The actual implementation with neccessary details and context are discussed in the section \ref{sec:implementation}. As an important part, performance of the simulation and related issues are the topic of section \ref{sec:performance}. We conclude the status quo of the project in section \ref{sec:conclusion}.

\subsection{Motivation}
\label{subsec:motivation}
Procedural terrain generation is an important part for a number of computer games \cite{generatedgames}. Usually, these games employ random generators to increase variety on a theme, such as a map layout. Indeed, in my subjective opinion a~player's experience is greatly enriched by variety, especially in the game environment. This comes with an apparent caveat that purely procedural generation may lack the sense of creativity, leading to mundanely repeating patterns \cite{daggerfall}. 

With the onset of newer technologies (e. g. increased GPU power), we are able to perform more computationally-intensive tasks. When it comes to the terrain generation, even a regular user without access to high-tier hardware can try more sophisticated alternatives to simpler algorithms. Arguably, more realistic worlds bring the feeling of familiarity to the experience. If we can create a~more realistic, yet still random map/world/neighbourhood, the possibilities are endless.
\newpage
Following thoughts are purely my personal view. As a life-long video games fan, I have always gravitated towards story-telling games, especially those taking place in an open world. Among these, I'd like to mention Baldur's Gate series, The Elder Scrolls series, Might \& Magic series, Fallout 4 and Mass Effect series. At the same time, I have been also drawn to grand building games taking place in complex worlds. Transport Tycoon or Caesar III and its modern re-implementation Augustus \cite{augustus} were a heavy influence, lately Factorio or Rimworld. Rimworld stands apart in its uniqueness, as it can be understood rather as a story generator than a game \cite{rimworld}. In large part, the idea of Driftworld came from the works of J. R. R. Tolkien and watching the 1997-2007 Stargate series - the series' take on mythology context in human societies in particular.

Epic stories build on cohesion. In this regard, countless debates take place about details. It takes a~great deal of time to create viable environment to match an idea for a story, if that story is told over long periods of time. Driftworld aspires to one thing: help create a platform in which stories can take place. First step is this project -- to create a rough map. 

\section{Spherical geometry \& topology}
\label{sec:geometry}
The most fundamental object with which Driftworld Tectonics works is a mesh of a sphere in the 3D Euclidean space. For simplicity, we assume the sphere is a unit sphere centered on the origin unless stated otherwise. Because of the spherical nature of the project, several (arguably) uncommon mathematical concepts are described in this section -- such as vertex sampling, triangulation, transformations or bounding volume hiearchies. Although the text follows almost a textbook-like mathematical structure, a lot of the formulations and conclusions lack correct proof. Some reasoning is made to carry a~point, but meticulous readers are left to their own devices.
\subsection{Unity coordinate system}
\label{subsec:unitycoords}
Unity uses a left-handed coordinate system with the \textit{x} axis pointing to the right, \textit{y} axis pointing upwards and \textit{z} axis pointing forward (see Figure \ref{fig:unity-coordinate}). This is reflected in the scenes -- nevertheless, the mathematical expressions of vectors themselves are identical to a standard right-handed coordinate system, i. e. the following holds for the basis:
$$\mathbf{e}_x \times \mathbf{e}_y = \mathbf{e}_z$$
All implementations must be aware of the fact that the cross product expressions do not distinguish between right-handed and left-handed. It is simply a matter of axes display, where visually 'switching' axes \textit{y} and \textit{z} alternates between left-handedness and right-handedness. In the left-handed coordinate system, right-hand rule of cross product shows the inverse final direction of the cross product.
\begin{figure}[ht]
\centering
\includegraphics[height=7cm]{unity-axes.png}
\includegraphics[height=7cm]{unity-rotation.png}
\caption{Unity coordinate system}
\label{fig:unity-coordinate}
\end{figure}

There are several ways to rotate points, vectors or whole transformations. For clarity, let us assume a 3-dimensional vector $\mathbf{u}$ that is to be rotated. We define a rotation unit vector $\mathbf{n}$ and an angle $\phi$ by which we rotate $\mathbf{u}$ so that $\mathbf{u}$ rotates by $\phi$ within a plane to which $\mathbf{n}$ is normal. We also assume that the rotation plane passes through the origin. Then from the perspective of a sundial (with $\mathbf{n}$ being the gnomon) $\mathbf{u}$ rotates \textit{clockwise} for positive $\phi$ (Figure \ref{fig:unity-coordinate}). This holds for all relative rotations.

\subsection{Sectional planes and great circles}
Sphere can have any number of sectional planes, i. e. planes that have some non-empty intersection with the sphere. Planes passing the center of the sphere will be called \textit{sectional central planes} (Figure \ref{fig:sectional-plane}). Any sectional central plane $\rho$ is characterized by some non-zero normal vector $\textbf{n}_\rho$ and for any point on the plane represented by their position vector $\mathbf{x}$ it holds that
$$\mathbf{n}_\rho\cdot\mathbf{x}=0$$
This is synonymous to the fact that any vector lying within a plane passing the origin is perpendicular to the normal vector of the plane. The dot product on the left side of the equality is also important because given a specific normal vector we can decide \textit{on which side} is any vector $\mathbf{x}$ outside the plane -- simply take the sign of the dot product, vector on the side of the normal vector will result in a positive dot product value with $\mathbf{n}_\rho$, negative otherwise.

An important object on the surface of a sphere is a great circle. It is any circle that shares its center and radius with the sphere (Figure \ref{fig:great-circle}). It is also the intersection of a plane passing the center of the sphere with its surface.
\begin{figure}[ht]
\centering
\begin{subfigure}{7cm}
\includegraphics[height=6cm]{sectional-plane.png}
\caption{central plane}
\label{fig:sectional-plane}
\end{subfigure}
\begin{subfigure}{7cm}
\includegraphics[height=6cm]{great-circle.png}
\caption{great circle}
\label{fig:great-circle}
\end{subfigure}
\caption{Sphere section by plane}
\label{fig:sectional-objects}
\end{figure}

\subsection{Spherical triangles}
\label{subsec:spherical-triangles}
Any three points on the surface of a sphere that do not lie on a single great circle form a~\textit{spherical triangle} (Figure \ref{fig:spherical-triangle}). This is the fundamental concept behind many of the computations in the project. However, strictly speaking, there are two triangles defined by such three points. The closure of the complement of any spherical triangle with respect to the sphere surface is also a spherical triangle, albeit one of the two is unintuitive as it is larger than half of the sphere surface area. To get around this, we construct somewhat narrower class of spherical triangles so that any three valid points define a triangle unambiguously.

\begin{figure}[ht]
\centering
\begin{subfigure}{8cm}
\includegraphics[height=6cm]{triangle-circumcircle.png}
\caption{with sectional plane and circumcircle}
\label{fig:triangle-circumcircle}
\end{subfigure}
\begin{subfigure}{8cm}
\includegraphics[height=6cm]{triangle.png}
\caption{planes intersections}
\label{fig:spherical-triangle-visual}
\end{subfigure}
\caption{Spherical triangle}
\label{fig:spherical-triangle}
\end{figure}

 We denote the surface of a unit sphere $\mathcal{S} =  \{\mathbf{x} \in \mathbb{R}^3: ||\mathbf{x}||=1\}$. Given a triplet of three linearly independent point vectors $(\mathbf{a},\mathbf{b},\mathbf{c})\in \mathcal{S}\times\mathcal{S}\times\mathcal{S}$ (called \textit{vertices})\footnote{Linear independence of unit vectors is equivalent to the condition that the vectors do not lie on a single great circle.}, we can construct a vector $\mathbf{n}_\lambda$ normal to some sectional plane $\lambda$ cutting off a spherical cap (Figure \ref{fig:triangle-circumcircle}):
 $$\mathbf{n}_\lambda=(\mathbf{b} - \mathbf{a})\times(\mathbf{c} - \mathbf{a})$$
$$\forall\mathbf{x}\in\lambda: \mathbf{n}_\lambda\cdot\mathbf{x}+d_\lambda=0$$
We can calculate $d_\lambda$ by assigning e. g. $\mathbf{x}=\mathbf{a}$ and solving the plane equation with respect to $d_\lambda$, but that will not be neccessary. We impose a further requirement $\mathbf{n}_\lambda\cdot\mathbf{a}>0$. This is not always true, in which case it can be ensured by swapping any two vertices in the triplet and recalculating $\mathbf{n}_\lambda$. This means that all three vertices and their circumcenter are all 'on one side' of the sphere. In Unity coordinate system, this also means that for an outside observer, the vertices are oriented \textit{clockwise} on the sphere surface.

Spherical circumcircle $l$ is a set of points on a sphere that has constant spherical distance from a single point $\mathbf{c}_\mathcal{T}\in\mathcal{S}$ called \textit{circumcenter}. Equivalently, we can substitute dot product for distance:
$$\exists t\in\mathbb{R}:(\forall\mathbf{x}\in l:\mathbf{c}_\mathcal{T}\cdot\mathbf{x}=t)$$
Since $\mathbf{n}_\lambda\cdot\mathbf{a}=\mathbf{n}_\lambda\cdot\mathbf{b}=\mathbf{n}_\lambda\cdot\mathbf{c}>0$, we know that some scalar multiple of $\mathbf{n}_\lambda$ is the circumcenter for the vertices. In fact, there is only one possible circumcircle for all three vertices, which is the intersection $\mathcal{S}\cap\lambda$. We easily find the circumcenter and the circumradius $r_l$ as\footnote{We have to keep in mind that on a unit sphere, central angle and spherical distance are identical, barring formal dimension.}
$$\mathbf{c}_\mathcal{T}=\frac{\mathbf{n}_\lambda}{||\mathbf{n}_\lambda||}, r_l=\arccos(\mathbf{c}_\mathcal{T}\cdot\mathbf{a})$$

Previous reasoning allows us now to test if some point $\mathbf{x}\in\mathcal{S}$ is inside a spherical triangle $\mathcal{T}\subset\mathcal{S}$ with clockwise-oriented vertices $(\mathbf{a}, \mathbf{b}, \mathbf{c})$.
Geometrically speaking, a spherical triangle is a region bounded by three arcs of great circles \cite{palmer}. We calculate three normal vectors:
$$\mathbf{n}_{\rho}=\mathbf{a}\times\mathbf{b},$$
$$\mathbf{n}_{\sigma}=\mathbf{b}\times\mathbf{c},$$
$$\mathbf{n}_{\tau}=\mathbf{c}\times\mathbf{a}.$$
These vectors define planes $\rho, \sigma, \tau$ so that
$$\rho=\{\mathbf{x}\in\mathbb{R}^3:\mathbf{n}_{\rho}\cdot\mathbf{x}=0\}$$
$$\sigma=\{\mathbf{x}\in\mathbb{R}^3:\mathbf{n}_{\sigma}\cdot\mathbf{x}=0\}$$
$$\tau=\{\mathbf{x}\in\mathbb{R}^3:\mathbf{n}_{\tau}\cdot\mathbf{x}=0\}$$
intersections $\rho\cap\mathcal{S}, \sigma\cap\mathcal{S}, \tau\cap\mathcal{S}$ are then great circles that always pass two of the vertices. Because of this, each one is divided by them into two arcs. There is only one triplet of arcs connected by the vertices that forms a meaningful region boundary on $\mathcal{S}$ (Figure \ref{fig:spherical-triangle-visual}). There are two such regions but we already bypassed this problem by ensuring orientation. We test the point $\mathbf{x}$ against following condition:
$$\mathcal{T}=\{\mathbf{x}\in\mathcal{S}:\mathbf{n}_{\rho}\cdot\mathbf{x}\ge 0, \mathbf{n}_{\sigma}\cdot\mathbf{x}\ge 0, \mathbf{n}_{\tau}\cdot\mathbf{x}\ge 0\}$$
This simply tells us that $\mathbf{x}$ is inside the spherical triangle $\mathcal{T}$ when it is on the surface of the sphere and also on one specific side of all three planes $\rho, \sigma, \tau$. This definition does not encompass all possible spherical triangles on a sphere, but it allows us to properly test the properties of any triangles used in reasonable spherical meshes.
\subsection{Vertex sampling}
Because of memory restrictions, sphere surface data is represented as a set of sampled points. We can identify these points as position vectors $\mathbf{u}_i$ from the global origin to sample points, resulting in a~sequence $U=\left(\mathbf{u}_i\right)_{i=0}^{N-1}, \mathbf{u}_i \in \mathcal{S}$. Samples are therefore three-dimensional normalized vectors. Driftworld uses spherical Fibonacci sampling \cite{keinert}. To get the sequence $U$, another sequence $F=\left(\mathbf{f}_i\right)_{i=0}^{N-1}$ is first computed, using the following definition:
$$\mathbf{f}_i=(\phi_i, z_i), \phi_i \in [0,2\pi), z_i \in (-1,1),$$
$$\phi_i = 2\pi\left[\frac{i}{\Phi}\right],$$
$$z_i = 1-\frac{2i+1}{N}.$$
$\left[x\right]$ denotes the fractional part of $x$, $\Phi$ is the golden ratio $\Phi=\frac{\sqrt{5}+1}{2}$. The values of $\mathbf{f}_i$ actually lie on a~spiral on the surface of a cylinder with the radius of 1 and the height of 2 \cite{keinert}. $U$ is finally obtained by mapping $\mathbf{f}_i$ values to $\mathcal{S}$:
$$\mathbf{u}_i = (\sin{(\arccos{(z_i)})}\cdot\cos{\phi_i}, z_i, \sin{(\arccos{(z_i)})}\cdot\sin{\phi_i})$$
Note that this mapping reflects Unity's axes orientation and the first and the last samples of $U$ do not fall exactly on the poles.
\subsection{Centroids, data values and barycentric interpolation}
Driftworld often makes computations for points inside sperical triangles - notably, it uses triangle centroids to evaluate triangle neighbours. Calculating these points is not a trivial procedure and although for a long time there have been methods to do so, it would be too resource-consuming when performed on a larger scale. To save computation time, we assume that all evaluated triangles are nearly planar, i. e. their \textit{triangle excess} is negligible (Legendre's Theorem\footnote{This theorem is also known as Saccheri-Legendre theorem.} \cite{todhunter}). We calculate the centroid of a triangle by simply normalizing the sum of its vertices (Figure \ref{fig:triangle-centroid}):
$$\mathbf{b}_\mathcal{T}=\frac{\mathbf{a} + \mathbf{b} + \mathbf{c}}{||\mathbf{a} + \mathbf{b} + \mathbf{c}||}$$
\begin{figure}[ht]
\centering
\begin{subfigure}{7cm}
\includegraphics[height=6cm]{triangle-centroid.png}
\caption{sectional triangle centroid}
\label{fig:triangle-centroid}
\end{subfigure}
\hspace*{1cm}
\begin{subfigure}{7cm}
\includegraphics[height=6cm]{triangle-barycentric.png}
\caption{barycentric coordinates}
\label{fig:triangle-barycentric}
\end{subfigure}
\caption{Centroid geometry}
\label{fig:centroid-geometry}
\end{figure}

To store crust data, we must assign values to points on the sphere. These values may be of different types or have different meaning. Formally, we denote a sequence of arbitrary sets $C=\left(V_i\right)_{i=0}^{n-1}$, where $n$ is the number of different values assigned to a point and each $V_i$ is a specific set of possible values. The system of all possible value combinations is then a cartesian product of these value sets:
$$V=\prod_{i=0}^{n-1}V_i$$
Stored data can then be defined as a map:
$$h: U\rightarrow V$$
$$h_i: U\rightarrow V_i$$
We store data only for sphere samples because of limited memory. Other values will be computed as needed using \textit{barycentric interpolation} \cite{scratchapixel}.\newpage

Now it comes to the following problem: how to compute values anywhere on $\mathcal{S}$? We are effectively looking for some domain extension, since $U\subset\mathcal{S}$:
$$h':\mathcal{S}\rightarrow V, \forall \mathbf{u}\in U:h'(\mathbf{u})=h(\mathbf{u})$$
$$h_i':\mathcal{S}\rightarrow V_i, \forall \mathbf{u}\in U:h_i'(\mathbf{u})=h_i(\mathbf{u})$$
Given some arbitrary point $\mathbf{x}\in\mathcal{S}$, we start with an assumption that $\mathbf{x}$ is found inside some spherical triangle $\mathcal{T}$ with negligible triangle excess and vertices $\{\mathbf{a}, \mathbf{b}, \mathbf{c}\} \subset U$ for which we already know the values of $h$. We would like $h'(\mathbf{x})$ to be computed 'fairly', i. e. the closer $\mathbf{x}$ is to some vertex, the more influence the vertex value should have on $h'(\mathbf{x})$. A good start might be in analogy with a political voting system based on area. If the population is homogeneous, any region vote is weighted by its area and transitionally, by its population.

Let $P$ be some point within a triangle $ABC$ (Figure \ref{fig:triangle-barycentric}). This point is represented by a point vector $\mathbf{p}\in\mathcal{S}$. As stated earlier, we assume all points lie nearly on the same plane. If we construct three triangles $PBC$, $APC$ and $APB$, the triangle $ABC$ will be divided into three regions, each corresponding to their oposite vertex of $ABC$. The closer $P$ is to any of the vertices, the larger the corresponding triangle area is. Total area sum of the three triangles is equal to the area of $ABC$. Therefore, we can use these triangle areas as weights for interpolating values at $P$ -- we only need to find the respective areas of $S_A, S_B, S_C$ and $S_{ABC}$. This can be done using cross product:
$$S_A=\frac{|(\mathbf{b}-\mathbf{p})\times(\mathbf{c}-\mathbf{p})|}{2}$$
$$S_B=\frac{|(\mathbf{c}-\mathbf{p})\times(\mathbf{a}-\mathbf{p})|}{2}$$
$$S_C=\frac{|(\mathbf{a}-\mathbf{p})\times(\mathbf{b}-\mathbf{p})|}{2}$$
$$S_{ABC}=\frac{|(\mathbf{b}-\mathbf{a})\times(\mathbf{c}-\mathbf{a})|}{2}$$
Since $S_A+S_B+S_C=S_{ABC}$, we can define normalized weights $u, v, w$ called \textit{barycentric coordinates}:
$$u=\frac{|(\mathbf{b}-\mathbf{p})\times(\mathbf{c}-\mathbf{p})|}{|(\mathbf{b}-\mathbf{a})\times(\mathbf{c}-\mathbf{a})|}$$
$$v=\frac{|(\mathbf{c}-\mathbf{p})\times(\mathbf{a}-\mathbf{p})|}{|(\mathbf{b}-\mathbf{a})\times(\mathbf{c}-\mathbf{a})|}$$
$$w=\frac{|(\mathbf{a}-\mathbf{p})\times(\mathbf{b}-\mathbf{p})|}{|(\mathbf{b}-\mathbf{a})\times(\mathbf{c}-\mathbf{a})|}$$
It is easy to confirm that $u+v+w=1$.

There are basically two types of values interpolated in the project -- real values and categories. Real value interpolation is straightforward:
$$h_i'(\mathbf{p})=uh_i(\mathbf{a})+vh_i(\mathbf{b})+wh_i(\mathbf{c})$$
Categories are simply assigned to $\mathbf{p}$ according to the largest weight:
$$u=\mbox{max}(\{u,v,w\})\Rightarrow h_j'(\mathbf{p})=h_j(\mathbf{a})$$
$$v=\mbox{max}(\{u,v,w\})\Rightarrow h_j'(\mathbf{p})=h_j(\mathbf{b})$$
$$w=\mbox{max}(\{u,v,w\})\Rightarrow h_j'(\mathbf{p})=h_j(\mathbf{c})$$
This effectively draws a Voronoi map according to categories \cite{voronoi}.
\subsection{Spherical mesh and Delaunay triangulation}
There is a basic sphere mesh, provided by Unity (Figure \ref{fig:unity-mesh}). It is like a detailed cubic mesh, projected onto a sphere. However, for finer terrain details, a much more detailed mesh is needed, preferably with uniform triangles (Figure \ref{fig:delaunay-mesh}).

\begin{figure}[ht]
\centering
\begin{subfigure}{7cm}
\includegraphics[height=7cm]{unity-mesh.png}
\caption{Unity sphere mesh}
\label{fig:unity-mesh}
\end{subfigure}
\hspace*{1cm}
\begin{subfigure}{7cm}
\includegraphics[height=7cm]{delaunay-mesh.png}
\caption{Delaunay mesh}
\label{fig:delaunay-mesh}
\end{subfigure}
\caption{Spherical meshes}
\label{fig:spherical-mesh}
\end{figure}
Definition and description of a 3D mesh is way beyond the scope and purpose of this document. In this context, it is simply an approximation of the sphere surface. All samples $U$ are vertices connected into triangles so that the whole sphere is covered by them without any gaps. For Driftworld, a set of prepared meshes is provided, created by \textit{Delaunay triangulation}\footnote{The triangulation algorithm is not a part of Driftworld. The meshes were actually constructed in a separate tool written in C++ and then added as data files to Driftworld Tectonics repository.} \cite{delaunay}.

When interpolating surface data such as elevation, it is important that reasonable samples are used for the interpolation. Calculating elevation in a mountain range from a triangle with vertices too far apart may result in meaningless artifacts. Furthermore, the earlier mentioned requirement that the triangles are nearly planar would be undermined by extreme spherical triangles which exhibit considerable excess. It stands to reason that triangles used in the mesh should be as regular as possible. This is the goal and result of a Delaunay triangulation. There is a number of algorithms performing the triangulation on a plane \cite{knuth}. Since a sphere has a closed mesh, an adaptation is needed.

Delaunay meshes for Driftworld use an algorithm which originally triangulates a set of random samples \cite{ma}. Because $U$ is ordered, the initial tetrahedron is somewhat difficult to construct, especially because of the requirements imposed on a spherical triangle -- in a large number of cases at least one triangle had a circumcircle larger than a great circle. For this reason, the initial structure was set to be a nearly regular octahedron with vertices assigned by a brute-force look-up. Other than that, the algorithm follows the article \cite{ma}.
\subsection{Collisions}
The tectonic model in Driftworld computes many interactions on a regular basis and these computations must be as efficient as possible. There are two basic collisions used for evaluating interactions -- a collision of two circles and a collision of two triangles. To clarify, in both cases we only need to answer the question whether the two objects have a non-empty intersection -- not to fully classify the intersection. Algorithms for both collisions are fairly simple and the spherical geometry actually helps in the case of triangle collisions.

The relative position of two circles $k,l\subset\mathcal{S}$ is governed by several parameters. Each circle has a~circumcenter $\mathbf{c} \in \mathcal{S}$ and a radius $r > 0$. We consider full circles to determine the intersection:
$$\forall\mathbf{x}\in\mathcal{S}:\arccos(\mathbf{x}\cdot\mathbf{c}_k) \le r_k\Rightarrow\mathbf{x}\in k$$
This means that there is only one case of relative circle position that has an empty intersection: disjoint circles. The case of one circle lying inside another has an intersection indentical to the inside circle. Three major cases of the relative positions of circles are seen in Figure \ref{fig:circle-collisions}.
\begin{figure}[ht]
\centering
\begin{subfigure}{7cm}
\includegraphics[height=7cm]{circle-intersection-a.png}
\caption{disjoint circles}
\label{fig:disjoint-circles}
\end{subfigure}\\
\begin{subfigure}{7cm}
\includegraphics[height=7cm]{circle-intersection-b.png}
\caption{circles intersecting at two points}
\label{fig:circles-intersecting-at-two-points}
\end{subfigure}
\hspace*{1cm}
\begin{subfigure}{7cm}
\includegraphics[height=7cm]{circle-intersection-c.png}
\caption{circle lying inside another}
\label{fig:circle-lying-inside-another}
\end{subfigure}
\caption{Relative positions of two circles}
\label{fig:circle-collisions}
\end{figure}

It is therefore easy to decide whether two circles collide or not - circles not colliding have a spherical distance larger than the sum of their radii:
$$k \cap l = \emptyset \Longleftrightarrow \arccos(\mathbf{c}_k\cdot\mathbf{c}_l)>r_k+r_l$$

In case of triangles the collision is more complex. There are four major cases of the relative position of two spherical triangles - three  similar to the case of circles and one specific (see Figure \ref{fig:triangle-collisions}). The second and the third case can be resolved by determining whether any point of one triangle lies within the other (see subsection \ref{subsec:spherical-triangles}). The first and the fourth require a more thorough test.

\begin{figure}[ht]
\centering
\begin{subfigure}{7cm}
\includegraphics[height=7cm]{triangle-intersection-a.png}
\caption{disjoint triangles}
\label{fig:disjoint-triangles}
\end{subfigure}
\hspace*{1cm}
\begin{subfigure}{7cm}
\includegraphics[height=7cm]{triangle-intersection-b.png}
\caption{triangles intersecting at two points}
\label{fig:triangles-intersecting-at-two-points}
\end{subfigure}\\
\begin{subfigure}{7cm}
\includegraphics[height=7cm]{triangle-intersection-c.png}
\caption{triangle lying inside another}
\label{fig:triangle-lying-inside-another}
\end{subfigure}
\hspace*{1cm}
\begin{subfigure}{7cm}
\includegraphics[height=7cm]{triangle-intersection-d.png}
\caption{Star of David}
\label{fig:star-of-david}
\end{subfigure}
\caption{Relative positions of two triangles}
\label{fig:triangle-collisions}
\end{figure}

If neither of the two triangles contain a vertex of the other one, we have to decide if any two edges intersect. Consider two edges defined by pairs of non-identical vertices $(\mathbf{a}_1, \mathbf{a}_2)$ and $(\mathbf{b}_1, \mathbf{b}_2)$. These define planes within which lie their respective great circles. The planes have normal vectors $\mathbf{a}_1 \times \mathbf{a}_2$ and $\mathbf{b}_1 \times \mathbf{b}_2$. Since the planes are central, Any intersection must be along the vector $(\mathbf{a}_1 \times \mathbf{a}_2)\times(\mathbf{b}_1 \times \mathbf{b}_2)$. There are two intersections in $\mathcal{S}$ and we consider the one maximizing the dot product with $\mathbf{a}_1$ (same hemisphere). We denote such intersection $\mathbf{i}$. The final test simply decides if $\mathbf{i}$ lies on both segments or is somewhere else on the great circles. This can be done by testing dot products, as $\mathbf{i}$ must be closer to both vertices than the distance of the vertices for both segments:
$$(\mathbf{i}\cdot\mathbf{a}_1 > \mathbf{a}_1\cdot\mathbf{a}_2) \land (\mathbf{i}\cdot\mathbf{a}_2 > \mathbf{a}_1\cdot\mathbf{a}_2)\land(\mathbf{i}\cdot\mathbf{b}_1 > \mathbf{b}_1\cdot\mathbf{b}_2) \land (\mathbf{i}\cdot\mathbf{b}_2 > \mathbf{b}_1\cdot\mathbf{b}_2)$$
If any two segments of the two triangles intersect, the triangles must by the sign analysis have non-empty intersection and therefore collide. If all tests are negative, the triangles are disjoint.
\subsection{Merging of spherical circles}
Because of the need to build bounding volume hiearchies (see subsection \ref{subsec:bvh}), there is a task of finding a~suitable circle $l$ which has the smallest area possible and which contains two other given circles $m,n$ (see Figure \ref{fig:circles-merged}). The center of $l$ must lie on a great circle $L$ containing both centers of the circles. This means we have to find a center and radius of $l$ so that it only touches either both of the two circles $m,n$ or one in case one circle is within the other. The algorithm falls in at least one of the following cases: concentric circles, circles with opposite centers, one circle contained in the other, circles intersecting at two points or disjoint circles. The goal is to determine which case has the deciding influence on the position of the center $\mathbf{c}_l$ and the radius $r_l$ except in the case of concentric circles, where $l$ is found easily. In case of concentric circles the solution is:
$$\mathbf{c}_l=\mathbf{c}_m, r_l=\mbox{max}(r_m, r_n)$$
Otherwise we find a suitable local basis in which we can easily parametrize all relevant points (Figure \ref{fig:merging-local-basis}). One circle center will be identical to the base vector:
$$\mathbf{e}_x'=\mathbf{c}_m$$
\begin{figure}[ht]
\centering
\begin{subfigure}{7cm}
\includegraphics[height=7cm]{merging-local-basis.png}
\caption{local basis}
\label{fig:merging-local-basis}
\end{subfigure}
\hspace*{1cm}
\begin{subfigure}{7cm}
\includegraphics[height=7cm]{circle-joining.png}
\caption{circles merged}
\label{fig:circles-merged}
\end{subfigure}
\caption{Merging of two circles}
\label{fig:merging-circles}
\end{figure}

Computation of the second base vector $\mathbf{e}_z'$ depends on whether the circles are directly opposite, i. e. they have centers opposite on the sphere. If so, it is any vector perpendicular to $\mathbf{e}_x'$ since the centers lie on inifinitely many great circles. If not, it it can be found with a double cross product:
$$\mathbf{e}_z'=\frac{(\mathbf{c}_m\times\mathbf{c}_n)\times\mathbf{c}_m}{||(\mathbf{c}_m\times\mathbf{c}_n)\times\mathbf{c}_m||}$$
There are three variables we need to compute now. First is the distance $d_{mn}$ between $\mathbf{c}_m$ and $\mathbf{c}_n$:
$$d_{mn}=\arccos(\mathbf{c}_m\cdot\mathbf{c}_n)$$
Second is the central angle $\Delta\phi$ of an arc running between $\mathbf{c}_m$ and the center of the merged circle. This allows us to compute the center as a linear combination of the base vectors $(\mathbf{e}_x', \mathbf{e}_z')$. Third is the actual radius $r_l$. However, we first need to see if one of the circles is contained within the other. If so, then one boundary is pushed by the encompassing circle:
$$-r_m > d_{mn} - r_n \Longrightarrow \Delta\phi=d_{mn}, r_l = r_n$$
$$r_m > d_{mn} +  r_n \Longrightarrow \Delta\phi=0, r_l = r_m$$
The first condition is for $n$ encompassing $m$, the second is the other way around. If neither is true, the computation is slightly more difficult:
$$\Delta\phi=\frac{d_{mn}-r_m+r_n}{2}$$
$$r_l=\frac{r_m + r_n + d_{mn}}{2}$$
Finally, we compute the center of $l$\footnote{Note that the value of $\Delta\phi$ can be negative. This is the case of clockwise-oriented arc from $\mathbf{c}_m$ in the local basis.}:
$$\mathbf{c}_l=\cos(\Delta\phi)\mathbf{e}_x' + \sin(\Delta\phi)\mathbf{e}_z'$$
\subsection{Texture mapping}
The system of overlays requires creating a texture every time the planet is to be rendered. Suppose we have a texture with resolution $w\times h$ that should have a color $b_{ij}$ from some color space $\mathcal{C}$ assigned to each of its points:
$$i\in\mathcal{I}=\{0,1,2,...,w-1\}$$
$$j\in\mathcal{J}=\{0,1,2,...,h-1\}$$
$$b:\mathcal{I}\times\mathcal{J}\rightarrow\mathcal{C}$$
We want $b_{ij}$ to reflect the data on the sphere. Let $b'$ be a map from the sphere surface, representing an overlay:
$$b': \mathcal{S}\rightarrow\mathcal{C}$$
We now have to find some map between the pixel indices and the sphere surface. This is very similar to the classical problem in cartography. Our choice will be an equirectangular projection because of its simplicity. We first compute the azimuthal and polar angles $\phi_i, \theta_j$ from $i,j$:
$$\phi_i = 2\pi\frac{\left(i+\frac{1}{2}\right)}{w}$$
$$\theta_j = \pi(1-\frac{j+\frac{1}{2}}{h})$$
\begin{figure}[ht]
\centering
\includegraphics[height=7cm]{texture-coordinates.png}
\caption{Cylinder rectangle around a sphere}
\label{fig:rectangle-sphere}
\end{figure}
This means the texture $y$ coordinate increases \textit{upwards} (to north) in the respective cylinder rectangle (Figure \ref{fig:rectangle-sphere}). From the angles it is simple to find the point on the sphere:
$$\mathbf{x}_{ij} = (\sin\theta_j\cos\phi_i,\cos\theta_j,\sin\theta_j\sin\phi_i)$$
Finally, we can compute $b_{ij}$:
$$b_{ij}=b'(\sin\theta_j\cos\phi_i,\cos\theta_j,\sin\theta_j\sin\phi_i)$$
This texture is subsequently used for rendering the surface, however, there is an inherent problem with uv mapping. Because of my insecurity about the math involved, the process will not be explained in this text and instead, the reader is encouraged to read a more educated article \cite{bgolus}.

There is perhaps a better way to deal with texture mapping, which was discussed in a Unity forum thread \cite{unityforum}. Cubemaps are quite possibly a more logical choice, as the texture information density needlesly increases towards extreme values of the $y$ coordinate.
\subsection{Unit dimensions}
Virtually all expressions throughout this section assumed a unit sphere for simplicity, but for rendering and parameter context some scaling is needed. For example, radius $R$ of the planet might be given in kilometers \cite{cortial}. Quantities such as this one describe planets in a~physical metric space with a~basic length unit of 1 m (and its metric prefixes). We consider the basic length unit in Unity scenes a 'Unity meter' $\mbox{u}$ and the corresponding space as Unity space. Given standard radius of a planet (in order of 1000 km), rendering in 1:1 scale from metric to Unity space is impractical. We set a scale convention $1\mbox{ u}=1000\mbox{ km}$. Because of the geological time scale, our basic unit of time will be $\mbox{My}$ (million years). Some useful conversions follow in table \ref{tab:unit-conversion}.

The planet radius expressed in Unity meters is used as a scaling parameter to transform quantities from Unity space to the unit sphere representation (denoted by $R_\mathcal{U}$). The simulation runs in the unit sphere representation and planet is rendered by scaling the vertices on the unit sphere by $R_\mathcal{U}$.

\paragraph{Example}Planet with a radius $R=6370$ km has a maximum plate speed\footnote{Because Driftworld Tectonics uses the unit sphere values, we will reference quantities declared in the original article~\cite{cortial} as primed.} $v_0'$ of 100 mm$\cdot$y$^{-1}$ and an average plate area $\mathcal{A}_0'$ of $25.5\times 10^{6}\mbox{ km}^2$ . We want to transform these values to a unit sphere representation. The scaling parameter is  $R_\mathcal{U}=6.37$ u. Maximum plate speed is transformed as:
$$v_0 = \frac{v_0'}{R_\mathcal{U}} =  \frac{100\mbox{ mm}\cdot\mbox{y}^{-1}}{6370\mbox{ km}}=\frac{0.1\mbox{ u}\cdot\mbox{My}^{-1}}{6.37\mbox{ u}}\approx 0.0157\mbox{ My}^{-1}$$
This allows us to identify any surface speed with an angular speed. The average area has two length dimensions, so to transform to a unit sphere, the expression is:
$$\mathcal{A}_0 = \frac{\mathcal{A}_0'}{R_\mathcal{U}^2} =  \frac{25.5\times 10^{6}\mbox{ km}^2}{40576900\mbox{ km}^2}= \frac{25.5\mbox{ u}^2}{40.5769\mbox{ u}^2}\approx 0.628$$
Note that the value is fraction-scaled to $4\pi$, which is the unit sphere surface area.
\begin{table}
\centering
\begin{tabular}{rr}
\textbf{Metric unit}&\textbf{Unity conversion}\\
\hline\\
1 km&0.001 u\\
1 km$^{-1}$&1000 u$^{-1}$\\
1 mm$\cdot$y$^{-1}$&$10^{-3}$ u$\cdot$My$^{-1}$
\end{tabular}
\caption{Unit conversion table}
\label{tab:unit-conversion}
\end{table}
\subsection{Vector noise on mesh}
\label{subsec:vector-noise-on-mesh}
Given a Delauney triangulation of a sphere, Driftworld uses a specific type of noise to randomize simple linear borders, such as between two tectonic plates. Because only whole mesh triangles are assigned to plates, a single noise value is assigned to each triangle. Because the goal of the randomizer is to shift border directions, the noise values are vectors. A single value $\mathbf{m}$ is obtained by taking a~random vector $\mathbf{s}\in\mathcal{S}$ and projecting it onto the tangent plane of the triangle centroid $\mathbf{c}$\footnote{This part is similar to the grid vector assignment when calculating 2D Perlin noise \cite{perlinnoise}, although with additional projection.}:
$$\mathbf{m} = \mathbf{s}-(\mathbf{s}\cdot\mathbf{c})\mathbf{c}$$
This way, the obtained random vectors should be uniformly distributed. Note that the vector lengths are from the interval $[0,1]$ and completely random. To avoid some of the border jitter, the noise should be low-frequency. Every triangle has three neighbours along its edges. This means that for every triangle we can average its noise vector with the neighbour noise vectors to filter higher frequency noise. We only need to ensure that the result is again within the tangent plane by projecting either the contributing vectors or the result itself. This summation can be repeated several time to adjust the filtering. 

The resulting low-frequency vector noise is driven by the number of averaging iterations. This is basically a 'smearing' of the triangle noise up to a certain radius. The total number of triangles in a~mesh influences the resulting noise pattern (detail) on the sphere. In Figure \ref{fig:vector-noise} the same number of iterations were used for meshes with 10,000 (\ref{fig:vector-noise-lowpoly}) and 500,000 (\ref{fig:vector-noise-highpoly}) triangles. It is clearly seen that the the granulation is finer on the sphere with more detailed mesh. Also, there is a noise pattern change along the lines where the mesh pattern changes. This is partially because each triangle is colored whole with a single color and the triangles are not perfectly regular.

It might be a good idea to revisit the concept of a vector noise on the sphere some time in the future. The concept is experimental and it is unclear if it has some unforeseen consequences for the plate border interactions.
\begin{figure}[ht]
\centering
\begin{subfigure}{7cm}
\includegraphics[height=7cm]{vector-noise-lowpoly.png}
\caption{low poly with mesh - whole triangles are coloured}
\label{fig:vector-noise-lowpoly}
\end{subfigure}
\hspace*{1cm}
\begin{subfigure}{7cm}
\includegraphics[height=7cm]{vector-noise-highpoly.png}
\caption{high poly without mesh - borders of mesh patterns visible}
\label{fig:vector-noise-highpoly}
\end{subfigure}
\caption{Vector noise representation - hue represents relative direction, saturation the length of a~vector, value is set to constant 1}
\label{fig:vector-noise}
\end{figure}

\section{Tectonic model}
\label{sec:tectonicmodel}
We now introduce the detailed description of the tectonic model used to create the planet crust. We discuss differences and similarities with respect to the original article and adapt the mechanisms to our unit sphere representation. Parameters that drive the model are summarized at the end of the section in Table \ref{tab:model-parameters-summary}. It should be stated that all parameters will be optimized for the number of vertex samples $N$ equal to 500,000.

The simplest description of the algorithm is that it creates some random crust partitioning into plates. These plates have randomized drifting parameters for moving. Then a certain number of tectonic steps is performed with a time step length $\delta t$ of 2 My and the result is a basic crust of the simulated planet. Between automated tectonic steps, user can force a few specific interactions (plate rifting, terrain smoothing) and change various global parameters to influence the simulation. The planet radius is set to $6370\mbox{ km}=6.37\mbox{ u}$.
\subsection{Workflow}
Basic crust is generated from a Delaunay triangulation of a unit sphere. The fresh crust is just a set of triangulated vertices, while each vertex is assigned some default crust data $h$. Each tectonic step consists of several substeps in a sequence. This sequence is firmly set because of implementation context and is depicted in Figure \ref{fig:tectonic-step-structure}. Short descriptions for the substeps follow.
\begin{figure}[ht]
\centering
\includegraphics[width=12cm]{tectonic-step-structure.png}
\caption{Tectonic step structure}
\label{fig:tectonic-step-structure}
\end{figure}
\begin{itemize}[\label={}]
\item[\textbf{Continental collisions}] If the plate drift would result in an overlap of two different continental (above sea level) areas of crust belonging to different plates, a continental collision is triggered, resulting in a~massive uplift of the lighter plate. The two plates in question are then merged into one. This is different from the orignal algorithm, which only attaches connected continental areas, called \textit{terranes}.
\item[\textbf{Plate drift}] Rotation transform of each plate along the sphere surface is updated by its respective angular speed.
\item[\textbf{Subduction uplift}] Overlapping areas of different plates cause uplift in the lighter plates. This process is known as \textit{subduction}. Note that overlapping continental areas have already been dealt with so this case should never occur during this step.
\item[\textbf{Continental erosion, ocean damping, sediment accretion}] This step updates elevation values to simulate erosion of crust above sea level, lowering of the ocean bottom for underwater crust and sediment accretion for the ocean crust below average ocean depth. This is questionable, as the original algorithm probably detects trench area for sediment filling.
\item[\textbf{Slab pull}] Subducting parts of plates tend to pull the plate towards them, altering the surface rotation. All vertices in subduction zones contribute to the rotation vectors of their plates.
\item[\textbf{Plate rifting}] Every tectonic step the largest plate has a chance to rift apart. Along some random linear border within the plate the vertices are assigned to two new plates with diverging velocities.
\item[\textbf{Crust aging}] Age of every crust vertex is updated by the length of the tectonic step.
\end{itemize}
\subsection{Crust \& plates}
The crust is defined as a set of surface vertices $U$, obtained by Fibonacci sampling. Its size is the number of samples $N$. Each vertex $\mathbf{u}_i\in U$ is assigned crust point data $h_i$. The tectonic plate system is an~equivalence system $\mathcal{P}$ of $U$ (so that every crust point belongs exactly to one plate). These equivalence classes should be connected through the mesh, but it is not a strict requirement (although it subtracts from the realism). Plates are the individual equivalence classes $\mathcal{P}_i\in\mathcal{P}$. If all vertices of a~mesh triangle belong to a~plate $\mathcal{P}_i$, the triangle is also said to belong to $\mathcal{P}_i$. A triangle only has three neighbours, each sharing one edge. If a triangle belonging to $\mathcal{P}_i$ has a neighbouring triangle which does not, it is called a \textit{border triangle}.

Each plate is also assigned: a centroid $\mathbf{c}_i$, rotation axis $\mathbf{w}_i$, plate angular speed $\omega_i$ and a transform $q_i$. The centroid is a vector calculated as the normalized sum of all vector representations of vertices belonging to the plate. If it cannot be normalized, a random vector is assigned. The rotation axis is a unit vector along the axis around which the plate drifts. The plate angular speed is self-explanatory. The transform is a quaternion representation of the relative rotation of the plate with respect to the original position. This is because the simulation does not actually move the plate vertices, only adjusts the plate transform to correctly calculate interactions. Moving the vertices would introduce serious problems with rendering.
\subsection{Crust data}
Crust data values included so far are: elevation, crust thickness, orogeny type and crust age. All crust points are strictly represented by unit vectors. The elevation values are information stored separately. Ocean crust are all crust points with negative elevation, continental crust points have a non-negative elevation. Crust thickness is a placeholder information for potential future updates. Orogeny type is represented by three categories: \textit{None}, \textit{Andean} and \textit{Himalayan}. The Andean type is a crust point elevated above the ocean level by subduction, the Himalayan type is a crust point that was influenced by continental collision. The None type is reserved for crust points not yet elevated by continental collision nor elevated above the ocean level by subduction. It does not exist in the original article, as the orogeny type is reserved for continental crust. The crust age is simple the time passed from the creation of the crust point. The original model also uses fold direction, which is not yet implemented, as I do not properly understand its purpose and mechanics.

The default crust point data for new points depends on whether the new point is continental or not. The only new continental points are created during the first partitioning (see Subsection \ref{subsec:plate-initialization}). Initial elevation is $z_{0t}=-0.004 \mbox{ u}$ for ocean crust and $z_{0c}=0.001 \mbox{ u}$ for continental crust. Crust thickness is always calculated from a~basic crust thickness value $e_0=0.01\mbox{ u}$ as $e=e_0+z$, where $z$ is the crust elevation. The initial orogeny type is None for all new ocean crust points and Andean for the initial continental crust. Initial crust age is universally equal to 0.
\subsection{Plate initialization}
\label{subsec:plate-initialization}
Partitioning of the crust into the initial set of plates  is governed by two parameters: the number of initial plates $N_\mathcal{P}$ and the probability of an initial plate being continental $p_C$. The initial number of plates is 40 and the probability of an initial plate to be continental is 0 for testing purposes. Before partitioning, vector noise is assigned to each triangle on the mesh with a noise averaging iterations parameter $n_{\mbox{sm}}$ of 4.

At first, $N_\mathcal{P}$ number of random points $\mathbf{c}$ (future \textit{centroids} of the plates) is distributed on the surface. Then all initial crust vertices are assigned to these points by their shortest distance on a unit sphere:
$$d(\mathbf{x},\mathbf{c})=\arccos(\mathbf{x}\cdot\mathbf{c})$$
All points that have the shortest distance to a certain centroid point belong to a single plate. This plate inherints the points and the centroid. When all points are distributed to their plates, each plate is then assigned a random rotation axis $\mathbf{w}$ and random non-negative angular speed $\omega$. The maximum plate angular speed is $v_0=0.0157\mbox{ My}^{-1}$. The symbol is unchanged to correspond with the original quantity. Finally, each plate is assigned a quaternion identity transform $q$. All crust points are assigned default data according to their plate. The probability of continental crust is evaluated on the plate level, so the initial plates all have the same elevation.

This kind of initialization basically creates a Voronoi diagram with perfectly straight lines (up to the triangle resolution). To simulate more realistic plate boundaries, vector noise is used. First we look for the triangles which have vertices from exactly two different plates. For each of these triangles we try to roll for probability equal to its noise vector magnitude. If the probability succeeds, we compute three dot products between the noise vector and each vector from the triangle barycenter to the vertex. The vertex with the maximum dot product is assigned to the plate of the vertex with the minimum dot product. This shifts the borders of the plates and is repeated $n_{\mbox{vb}}=4$ times. This parameter is the number of Voronoi border shift iterations. An example of the resulting partitioning can be seen in Figure \ref{fig:initial-plates}.
\begin{figure}[ht]
\centering
\begin{subfigure}{7cm}
\includegraphics[height=7cm]{plate-initialization.png}
\caption{initial plates}
\label{fig:initial-plates}
\end{subfigure}
\hspace*{1cm}
\begin{subfigure}{7cm}
\includegraphics[height=7cm]{plate-drift.png}
\caption{plates drifting}
\label{fig:plates-drifting}
\end{subfigure}
\caption{Crust partitioning}
\label{fig:crust-partitioning}
\end{figure}
\subsection{Plate overlaps}
Tectonic interactions require the concept of plate density. For example, denser ocean plates are subducted under continental plates. Because our model does not have a clear designation of a plate as ocean or continental (any plate can have ocean or continental crust), we evaluate plates by a weighted sum of their vertices. Each plate is assigned a score equal to:
$$\mbox{score}=100\times\mbox{number of continental crust points}-\mbox{number of ocean crust points}$$
The plates are then ranked by the highest score. The rank decides which plate 'goes under' when two plates overlap (the one with a lower score). This actually creates an irreflexive, antisymmetric and transitive relation on the set of plates.

This ranking system is a gross simplification. As many simplifications, though, it makes certain decisions and calculations much easier. The plate ranks have to be recalculated every time an interaction requires them, as the scores change both with crust elevation and changes in crust point assignments to plates. An example might be when we need to know which of the overlapping plates defines a crust point elevation on the surface.
\subsection{Plate drift}
The main reason for tectonic interactions is the tectonic drift. Plates move constantly, causing collisions, subduction etc. To model the drift of a plate, during every tectonic step the plate transform is adjusted
by multiplication of the transform by a quaternion representing a rotation around the axis $\mathbf{w}$ by the angle of $\Delta\phi=\omega\delta t$. This keeps the information about current crust points locations. The result of a one step drift from the initial position can be seen in Figure \ref{fig:plates-drifting}. We can see here that the plates move as individual rigid bodies.
\subsection{Ocean crust generation \& crust resampling}
Moving rigid plates necessarily create gaps on the surface. In reality, this 'empty' space is filled with new crust drifting from ocean ridges between the plates (see Figure \ref{fig:resample-mesh}). We use the original model, only slightly simplified. We can interpolate crust data at any point on the surface which is not in any triangle belonging to a plate. We compute two distances to two nearest plates (nearest vertices belonging to two different plates) $d_1$ and $d_2$ (1 being the absolute shortest) and assume that the point is approximately on the direct line between the nearest points. We also assume that the ridge is directly in the middle of the line. This is not true in reality, but makes it simple to use the original algorithm easily. We compute the ridge and plate elevation contributions and combine them as per Cortial et al.
\begin{figure}[ht]
\centering
\begin{subfigure}{7cm}
\includegraphics[height=7cm]{resample-drift.png}
\caption{diverging plates}
\label{fig:resample-drift}
\end{subfigure}
\hspace*{1cm}
\begin{subfigure}{7cm}
\includegraphics[height=7cm]{resample-ridges.png}
\caption{generated ridges}
\label{fig:resample-ridges}
\end{subfigure}
\caption{Ocean crust generation}
\label{fig:resample-mesh}
\end{figure}

The ridge function profile uses three parameters: the highest ocean ridge elevation $z_r$, the abyssal plains elevation $z_a$ and the ocean ridge elevation falloff $\sigma_r$. The values of these parameters are:
$$z_r=-1\mbox{ km}=-0.001\mbox{ u}$$
$$z_a=-6\mbox{ km}=-0.006\mbox{ u}$$
$$\sigma_r=0.05\mbox{ u}$$
The ridge function profile is a function of a variable $d_\Gamma$ which is the distance to the ocean ridge. The function profile is chosen as:
$$z_\Gamma=(z_r-z_a)2^{\frac{d_\Gamma}{\sigma_r}}+z_a$$
The original function profile is not specifically described and may be more complex.
Because of our assumptions we can calculate the ridge function profile variable that is the distance to the ridge as:
$$d_\Gamma=\frac{d_2-d_1}{2}$$
The orogeny type is universally filled as None for points in the surface voids. The crust age is computed as the scaling parameter $\alpha=\frac{d_\Gamma}{d_\Gamma+d_1}$ multiplied by the total time for which the plates have been diverging (since last they were in close contact). This makes the new ocean crust gradually older the further it is from the ridge. Finally, any new ocean crust point is assigned to the nearest plate.

For simulation running for many steps it is vital that we periodically resample the surface to fill the gaps made by diverging plates and to resolve overlapping plates. As per Cortial et al., it is recommended to resample the surface every 10th-60th step. Our model forces resampling on several occasions, namely continental collision. The resampling simply means to interpolate surface data onto the original mesh from the current surface data defined on a mesh broken by drifting plates and interactions. For each initial mesh vertex, we test if it is found on a plate with the highest rank possible. If so, we perform barycentric interpolation from a triangle within which the vertex currently resides. If no plate is found, we create a new ocean crust point.
\begin{table}[h]
\centering
\begin{tabular}{cccc}
\textbf{Symbol}&\textbf{Description}&\textbf{Original value}&\textbf{Model value}\\
\hline
$N$&Number of mesh vertices&-&500,000\\
$\delta t$&Tectonic time step&2 My&2 My\\
$R$&Planet radius&6,378 km&6.37 u\\
$z_{0t}$&Initial ocean elevation&-&-0.004 u\\
$z_{0c}$&Initial continental elevation&-&0.001 u\\
$e_0$&Basic crust thickness&-&0.01 u\\
$N_\mathcal{P}$&Initial number of plates&-&40\\
$p_C$&Initial continental plate probability&0.3&0\\
$v_0$&Maximum plate speed&100 mm$\cdot$y$^{-1}$&$0.0157\mbox{ My}^{-1}$\\
$n_{\mbox{sm}}$&noise averaging iterations&-&4\\
$n_{\mbox{vb}}$&Voronoi border shift iterations&-&4\\
$z_r$&Highest ocean ridge elevation&$-1\mbox{ km}$&$-0.001\mbox{ u}$\\
$z_a$&Abyssal plains elevation&-6\mbox{ km}&$-0.006\mbox{ u}$\\
$\sigma_r$&Ocean ridge elevation falloff&-&0.05\mbox{ u}\\
\end{tabular}
\caption{Model parameters summary}
\label{tab:model-parameters-summary}
\end{table}
\section{Implementation \& data model}
\label{sec:implementation}
\subsection{Rendering}
\label{subsec:rendering}
\subsection{GPU Computing}
\subsection{Bounding volume hiearchy}
\label{subsec:bvh}
When dealing with interactions of objects consisting of many triangles, such as parts of a mesh or meshes, testing every triangle against each other is very inefficient, leading to algorithmic complexity of $\mathcal{O}(n^2)$. For this, Driftworld implements \textit{bounding volume hiearchy} (BVH) over the triangles of the sphere mesh \cite{sulaiman}. Bounding volume (BV) is a simple object that contains a primitive or a group of primitives. Its purpose is to simplify collision tests -- for this, bounding volumes are structured in a~hiearchy. The most common BVH is a tree -- its leaves are bounding volumes of individual primitives. Parent nodes of such a tree are bounding volumes created by convenient merging of their children node BVs. The root of the tree is then some whole object. For instance, it could be a bounding volume of a mesh, while the leaves would be the singular triangles or their bounding volumes. There are multiple types of various bounding volumes and multiple types of constructed BV trees.

For our purpose, we construct a BVH for the triangles inside some region of the sphere triangulation. Our bounding volume is a spherical circle, the tree is a binary tree of merged circles. This choice makes sense because the triangles only move on the surface of the sphere and therefore we do not need spatial bounding volumes. Mathematically
\subsection{Use}

\section{Performance \& problems}
\label{sec:performance}
The project was developed and tested on an Intel Core i5-10600 clocked at 3.3~GHz with 16~GB of RAM and NVIDIA RTX 2060 SUPER. The simulations were tested in the 500,000 data samples and 60,000 render samples configuration to achieve maximum resolution. There are significant differences in the performance of Driftworld Tectonics and the procedural method by Cortial et al. The 500,000 data sample does not allow for smooth interactivity, possibly due to the dependence on internal Unity mechanisms and limited optimization. A~single tectonic step takes between 3-10 seconds, depending on continental collisions and rifting events. Arguably, a~credible landmass layout can be achieved in around 60 steps when using laplacian smoothing and/or manual elevation. Datawise, the output is robust enough to not cause serious incorrigible artifacts. An example of a fully formed continent after 59 steps using standard configuration (see summaries of the model and implementation, tables \ref{tab:model-parameters-summary} and \ref{tab:implementation-parameters-summary}) is seen in Figure~\ref{fig:output-example}.
\begin{figure}[ht]
\centering
\includegraphics[height=13cm]{performance-continent.png}
\caption{Continental ladmass - example of the simulation output}
\label{fig:output-example}
\end{figure}

Configurations with 60,000 data samples or less are much faster, although they do suffer from decreased resolution, especially around crust borders. This, however, should be mitigated by proper terrain amplification.

The initial number of tectonic plates has a~performance consequences for the simulations. Higher number of plates implies larger subduction fronts and thus more landmass is created, while lower number of plates causes landmass to actually decrease in size. It is therefore convenient to keep the number of plates as high as possible.

Driftworld Tectonics performance has several issues that can possibly be addressed by further optimalization. Following subsections describe the most prominent issues and how to counter them.

\subsection{Interpolation artifacts}
Layer data is interpolated many times during a simulation and on certain conditions the results are imprecise. This is mostly prominent on perfectly flat surfaces and along crust plate borders. The artifacts show as peaks on the surface and have most frequently the size of a single sample point. These very quickly disappear when actual tectonic steps are performed, as the terrain is no longer flat and any artifacts are hidden within the complexity of the terrain (see Figure \ref{fig:issue-interpolation}).
\begin{figure}[ht!]
\centering
\includegraphics[height=9cm]{issue-interpolation.png}
\caption{Flat surface peaks - interpolation imperfections}
\label{fig:issue-interpolation}
\end{figure}

\subsection{Missing texture data}
Calculations of overlay textures relies heavily on interpolation. Any falsely negative triangle tests because of precision errors result in 'dead' pixels in the texture. This is purely a rendering issue and does not pose a problem for the simulation (see Figure \ref{fig:issue-dead-pixels}).
\begin{figure}[ht!]
\centering
\includegraphics[height=9cm]{performance-B.png}
\caption{Missing texture data}
\label{fig:issue-dead-pixels}
\end{figure}

\subsection{Smoothing borders}
As a direct result of a terrain smoothing, borders are somewhat simplified, which may cause an unnatural look, especially when it is overused. This can be improved by adding tectonic steps without smoothing but sometimes it is better to load a previous state of the simulation and look for another way (see Figure \ref{fig:issue-blobs}).
\begin{figure}[ht!]
\centering
\includegraphics[height=9cm]{performance-A.png}
\caption{Unnatural smoothing of continental shores}
\label{fig:issue-blobs}
\end{figure}

\subsection{Crust furrows}
During a simulation after 20-30 steps, sometimes a crust artifact occurs where the terrain is creased along one direction (see Figure \ref{fig:crust-furrows}). This is not a~result of crust folding as folding is not implemented. This creasing is usually accompanied by extreme elevation gradients. The reasons for these occurences are not very well understood yet but the artifacts can be corrected to a~degree using Laplacian smoothing. The result is usually a group of elongated islands.
\begin{figure}[ht!]
\centering
\includegraphics[height=9cm]{issue-furrows.png}
\caption{Crust furrows}
\label{fig:crust-furrows}
\end{figure}

\subsection{Compute shader crashes}
For high sample count simulations (500,000 data samples), computations on the crust layer stretch GPU compute shaders to their limits. For crust plates number exceeding 30, tectonic steps sometimes cause crashes on compute shader dispatches. This is the main reason why the recommended initial number of plates is 20. For lower number of samples, 40 plates is a~viable option and should be used to create more interesting crust formations. The lower number of plates can be counter-weighted by applying manual elevation increments, such as in the example in Figure  \ref{fig:output-example}. It is possible that a~better GPU might not have the same issues, but this was not tested.
\subsection{Memory leaks}
The simulations of 500,000 data samples are somewhat RAM-intensive. After many tectonic and rendering steps, the simulation may continuously increase its used operational memory. This is obviously due to memory leaks, which may have two probable causes: constant resampling of the crust or memory lost due to texture reassignments. Until the problem is properly addressed, it is recommended that users periodically save their simulations to files and then load them to restarted Unity editor scenes. The rate of the memory increase is small but steady -- 100 tectonic steps with renders seem to increase the required memory by about 5 GB at 500,000 data samples.
\section{Conclusion}
\label{sec:conclusion}
The promise of Driftworld Tectonics is to create interesting planet surfaces either for rendering or further development -- a~goal which I believe it has achieved. The simulations loosely follow tectonic processes present on Earth. Since the project is an attempted implementation of the broadly described tectonic method of Cortial et al., it should provide similar features and advantages. Indeed, many of the features seem to be present (at least in my opinion). However, it should be noted that the work presented by Cortial et al. is much more sophisticated. Driftworld Tectonics, however, offers a~detailed description of its every aspect, which may inspire further improvements, optimizations and extensions. The project is meant to be as modular as possible so that changes would not break the project structure too badly.

The lack of terrain amplification is intentional to keep the project somewhat simple. All relevant data is available in the output files, so any subsequent amplification procedure may take place. It stands to reason that folding direction should be implemented in this stage of simulation -- hopefully, this will soon be the case.

There is still much work to be done. Issues that have come up during the development of Driftworld need to be addressed so that any simulation can be reliable and smoothly operating.

It is my hope that this project inspires content creators and developers to either improve on present tectonic approximation projects or make use of Driftworld Tectonics or other available software to create worlds and stories told on these worlds.
\subsection{Continuation of work}
Driftworld Tectonics is only the first step. The real ambition is to create rich worlds with water, weather, winter or spring, plants and animals, and ultimately, people. Each part of this ambition is a~huge project on its own, it has different requirements, different ideas and approaches. At this moment, I can hardly imagine the complexity of these projects and how monumental the effort will be to undertake them. I do believe that every part of this ambition has some value on its own.

For the immediate future, a~climate model should be created on the simulation output. Calculating heat distribution during a year and the hydrosphere model seems like a logical next step. Then, amplification of the terrain is necessary to obtain any kind of presentable planet.

Any notion of a~biosphere depends on the previous steps and in turn, influences both the climate and the hydrosphere as a~whole. A~stable model should follow where deserts have a~reason and forests do not thrive without rain. The goal is not to create true simulations but to bring believable biomes.

The final step -- people. This unfortunately seems so far that all ambition -- culture, language, history, agriculture, war or infrastructure -- is a~vague, shapeless dream, so far only in the imagination of a~writer. Maybe after a~few years, some parts of this ambition will take more solid shape so that anyone can be surprised to hear an exotic language spoken on a~tropical island in the middle of an ocean.
\newpage
\begin{thebibliography}{9}
\bibitem{cortial}
Cortial, Y., Peytavie, A., Galin, E. and Guérin, E. (2019), Procedural Tectonic Planets. \textit{Computer Graphics Forum}, 38: 1-11. https://doi.org/10.1111/cgf.13614
\bibitem{meister}
D. Meister and J. Bittner, "Parallel Locally-Ordered Clustering for Bounding Volume Hierarchy Construction," in \textit{IEEE Transactions on Visualization and Computer Graphics}, vol. 24, no. 3, pp. 1345-1353, 1 March 2018, doi: 10.1109/TVCG.2017.2669983.
\bibitem{generatedgames}
Wikipedia contributors. (2022, March 20). List of games using procedural generation. In \textit{Wikipedia, The Free Encyclopedia}. Retrieved 06:20, May 19, 2022, from \url{https://en.wikipedia.org/w/index.php?title=List\_of\_games\_using\_procedural\_generation\&oldid=1078237416}
\bibitem{daggerfall}
Brogan, J. (2016, October 5). \textit{The Daggerfall Paradox}. SLATE. Retrieved May 19, 2022, from\\ \url{https://slate.com/technology/2016/10/the-paradox-of-procedurally-generated-video-games.html}
\bibitem{augustus}
Keriew, \textit{Augustus}, (2021), GitHub repository, \url{https://github.com/Keriew/augustus}
\bibitem{rimworld}
Game Developer Conference [GDC]. (2019). \textit{RimWorld: Contrarian, Ridiculous, and Impossible Game Design Methods} [Video]. YouTube. https://www.youtube.com/watch?v=VdqhHKjepiE
\bibitem{keinert}
Keinert, Benjamin  \& Innmann, Matthias \& Sänger, Michael \& Stamminger, Marc. (2015). Spherical Fibonacci Mapping. \textit{ACM Transactions on Graphics}. 34. 1-7. 10.1145/2816795.2818131. 
\bibitem{palmer}
Palmer, C. I., \& Leigh, C. W. (1934). \textit{Plane and spherical trigonometry}. New York: McGraw-Hill Book Company, Inc. 
\end{thebibliography}
\newpage
\listoffigures
\listoftables
\newpage
\appendix
\input{sections/TemplateDataFileStructure}
\section{Save datafile structure - version 1}
\label{sec:save-datafile-structure}
First iteration of the simulation data file is, like the template data file, a~structured byte stream with a header, ordered in little-endian.

\begin{enumerate}
\item int value - denoting the length $l$ of the header ($1\times$4 B)
\item ASCII char array of length $l$ ('DRIFTWORLD TECTONICS' in the current version) ($l\times$1 B)
\item int value - save file version (1) ($1\times$4 B)
\item int value as bool - whether tectonic plates are present ($1\times$4 B)
\item float value - planet radius ($1\times$4 B)
\item uint value - index into internal RNG state ($1\times$4 B)
\item uint array representing the internal RNG state ($624\times$4 B)
\item if tectonics are present:
\begin{enumerate}
\item int value - number of tectonic steps taken without resampling ($1\times$4 B)
\item int value - total number of tectonic steps taken ($1\times$4 B)
\end{enumerate}
\item int value - denoting $n_d$ number of data layer vertices ($1\times$4 B)
\item crust and data vertex values ($n_d\times$?):
\begin{enumerate}
\item if tectonics are present:
\begin{enumerate}
\item float value - x coordinate of the current crust vertex ($1\times$4 B)
\item float value - y coordinate of the current crust vertex ($1\times$4 B)\footnote{Note that the y coordinate actually corresponds to the z coordinate in a~tight-handed system.}
\item float value - z coordinate of the current crust vertex ($1\times$4 B)
\item float value - current crust vertex elevation ($1\times$4 B)
\item float value - current crust vertex thickness ($1\times$4 B)
\item int value - current crust vertex plate index within the plate array ($1\times$4 B)
\item float value - current crust vertex age ($1\times$4 B)
\item int value as enum - current crust vertex orogeny ($1\times$4 B)
\end{enumerate}
\item float value - x coordinate of the current data vertex ($1\times$4 B)
\item float value - y coordinate of the current data vertex ($1\times$4 B)
\item float value - z coordinate of the current data vertex ($1\times$4 B)
\item float value - current data vertex elevation ($1\times$4 B)
\item float value - current data vertex thickness ($1\times$4 B)
\item int value -  current data vertex plate index within the plate array ($1\times$4 B)
\item float value - current data vertex age ($1\times$4 B)
\item int value as enum - current data vertex orogeny ($1\times$4 B)
\item int value - current data vertex neighbours count $k_n$ ($1\times$4 B):
\item int array of current data vertex neighbour indices within the data vertex array ($k_n\times$4 B)
\item int value - current data vertex corresponding triangle count $k_t$ ($1\times$4 B):
\item int array of current data vertex corresponding triangle indices within the data triangle array ($k_t\times$4 B)
\end{enumerate}
\item int value - denoting $m_d$ number of data layer triangles ($1\times$4 B)
\item crust and data triangle values, vector noise ($m_d\times$?):
\begin{enumerate}
\item if tectonics are present:
\begin{enumerate}
\item int value - A vertex index of the current crust triangle ($1\times$4 B)
\item int value - B vertex index of the current crust triangle ($1\times$4 B)
\item int value - C vertex index of the current crust triangle ($1\times$4 B)
\item int value - first neighboring triangle index of the current crust triangle ($1\times$4 B)
\item int value - second neighboring triangle index of the current crust triangle ($1\times$4 B)
\item int value - third neighboring triangle index of the current crust triangle ($1\times$4 B)
\end{enumerate}
\item int value - A vertex index of the current data triangle ($1\times$4 B)
\item int value - B vertex index of the current data triangle ($1\times$4 B)
\item int value - C vertex index of the current data triangle ($1\times$4 B)
\item int value - first neighboring triangle index of the current data triangle ($1\times$4 B)
\item int value - second neighboring triangle index of the current data triangle ($1\times$4 B)
\item int value - third neighboring triangle index of the current data triangle ($1\times$4 B)
\item float value - vector noise x coordinate of the current data triangle ($1\times$4 B)
\item float value - vector noise y coordinate of the current data triangle ($1\times$4 B)
\item float value - vector noise z coordinate of the current data triangle ($1\times$4 B)
\end{enumerate}
\item if tectonics are present:
\begin{enumerate}
\item int value - denoting the number of tectonic plates plates array $n_p$ ($1\times$4 B)
\item plates array ($n_p\times$?):
\begin{enumerate}
\item float value - current plate rotation axis x coordinate ($1\times$4 B)
\item float value - current plate rotation axis y coordinate ($1\times$4 B)
\item float value - current plate rotation axis z coordinate ($1\times$4 B)
\item float value - current plate angular speed ($1\times$4 B)
\item float value - current plate quaternion transform x coordinate ($1\times$4 B)
\item float value - current plate quaternion transform y coordinate ($1\times$4 B)
\item float value - current plate quaternion transform z coordinate ($1\times$4 B)
\item float value - current plate quaternion transform w coordinate ($1\times$4 B)
\item float value - current plate quaternion centroid x coordinate ($1\times$4 B)
\item float value - current plate quaternion centroid y coordinate ($1\times$4 B)
\item float value - current plate quaternion centroid z coordinate ($1\times$4 B)
\end{enumerate}
\end{enumerate}
\item int value - denoting $n_r$ number of render layer vertices ($1\times$4 B)
\item render vertex values ($n_r\times$?):
\begin{enumerate}
\item float value - x coordinate of the current render vertex ($1\times$4 B)
\item float value - y coordinate of the current render vertex ($1\times$4 B)
\item float value - z coordinate of the current render vertex ($1\times$4 B)
\item float value - current render vertex elevation ($1\times$4 B)
\item float value - current render vertex thickness ($1\times$4 B)
\item int value -  current render vertex plate index within the plate array ($1\times$4 B)
\item float value - current render vertex age ($1\times$4 B)
\item int value as enum - current render vertex orogeny ($1\times$4 B)
\item int value - current render vertex neighbours count $k_{rn}$ ($1\times$4 B):
\item int array of current render vertex neighbour indices within the render vertex array ($k_{rn}\times$4 B)
\item int value - current render vertex corresponding triangle count $k_{rt}$ ($1\times$4 B):
\item int array of current render vertex corresponding triangle indices within the render triangle array ($k_{rt}\times$4 B)
\end{enumerate}
\item int value - denoting $m_{rd}$ number of data layer triangles ($1\times$4 B)
\item crust and data triangle values, vector noise ($m_{rd}\times$?):
\begin{enumerate}
\item int value - A vertex index of the current render triangle ($1\times$4 B)
\item int value - B vertex index of the current render triangle ($1\times$4 B)
\item int value - C vertex index of the current render triangle ($1\times$4 B)
\item int value - first neighboring triangle index of the current render triangle ($1\times$4 B)
\item int value - second neighboring triangle index of the current render triangle ($1\times$4 B)
\item int value - third neighboring triangle index of the current render triangle ($1\times$4 B)
\end{enumerate}
\end{enumerate}


\end{document}
